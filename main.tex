\documentclass[a4paper, article, oneside, UKenglish]{memoir}


%% Title page
\usepackage{projectfp} % [MAT2000], [MAT2500], [MEK3200] or [STK-MAT2011]


%% Encoding
\usepackage[utf8]{inputenx} % Source code
\usepackage[T1]{fontenc}    % PDF


%% Fonts and typography
\usepackage{lmodern}           % Latin Modern Roman
\usepackage[scaled]{beramono}  % Bera Mono (Bitstream Vera Sans Mono)
\renewcommand{\sfdefault}{phv} % Helvetica
\usepackage[final]{microtype}  % Improved typography
\renewcommand{\abstractnamefont}{\sffamily\bfseries}                 % Abstract
\renewcommand*{\chaptitlefont}{\Large\bfseries\sffamily\raggedright} % Chapter
\setsecheadstyle{\large\bfseries\sffamily\raggedright}               % Section
\setsubsecheadstyle{\large\bfseries\sffamily\raggedright}            % Subsection
\setsubsubsecheadstyle{\normalsize\bfseries\sffamily\raggedright}    % Subsubsection
\setparaheadstyle{\normalsize\bfseries\sffamily\raggedright}         % Paragraph
\setsubparaheadstyle{\normalsize\bfseries\sffamily\raggedright}      % Subparagraph

%% Mathematics
\usepackage{amssymb}   % Extra symbols
\usepackage{amsthm}    % Theorem-like environments
\usepackage{thmtools}  % Theorem-like environments
\usepackage{mathtools} % Fonts and environments for mathematical formuale
\usepackage{mathrsfs}  % Script font with \mathscr{}


%% Miscellaneous
\usepackage{graphicx}  % Tool for images
\graphicspath{{figures/}}
\usepackage{babel}     % Automatic translations
\usepackage{csquotes}  % Quotes
\usepackage{textcomp}  % Extra symbols
\usepackage{listings}  % Typesetting code
\lstset{basicstyle = \ttfamily, frame = tb}


%% Bibliography
\usepackage{mathscinet}
\usepackage[backend    = biber,
            sortcites  = true,
            giveninits = true,
            doi        = false,
            isbn       = false,
            url        = false,
            sortlocale = nb_NO,
            style      = alphabetic]{biblatex}
\DeclareNameAlias{sortname}{family-given}
\DeclareNameAlias{default}{family-given}
\DeclareFieldFormat[article]{volume}{\bibstring{jourvol}\addnbspace#1}
\DeclareFieldFormat[article]{number}{\bibstring{number}\addnbspace#1}
\renewbibmacro*{volume+number+eid}
{
    \printfield{volume}
    \setunit{\addcomma\space}
    \printfield{number}
    \setunit{\addcomma\space}
    \printfield{eid}
}
\addbibresource{bibliography.bib}


%% Cross references
\usepackage{varioref}
\usepackage[pdfusetitle]{hyperref}
\urlstyle{sf}
\usepackage[nameinlink, capitalize, noabbrev]{cleveref}
\crefname{chapter}{Section}{Sections}


%% Theorem-like environments
\declaretheorem[style = plain, numberwithin = chapter]{theorem}
\declaretheorem[style = plain,      sibling = theorem]{corollary}
\declaretheorem[style = plain,      sibling = theorem]{lemma}
\declaretheorem[style = plain,      sibling = theorem]{proposition}
\declaretheorem[style = definition, sibling = theorem]{definition}
\declaretheorem[style = definition, sibling = theorem]{example}
\declaretheorem[style = remark,    numbered = no]{remark}


%% Delimiters
\DeclarePairedDelimiter{\p}{\lparen}{\rparen}   % Parenthesis
\DeclarePairedDelimiter{\set}{\lbrace}{\rbrace} % Set
\DeclarePairedDelimiter{\abs}{\lvert}{\rvert}   % Absolute value
\DeclarePairedDelimiter{\norm}{\lVert}{\rVert}  % Norm


%% Operators
\newcommand{\diff}{\mathop{}\!\mathrm{d}}
\DeclareMathOperator{\im}{im}
\DeclareMathOperator{\rank}{rank}
\DeclareMathOperator{\E}{E}
\DeclareMathOperator{\Var}{Var}
\DeclareMathOperator{\Cov}{Cov}


%% New commands for sets
\newcommand{\N}{\mathbb{N}}   % Natural numbers
\newcommand{\Z}{\mathbb{Z}}   % Integers
\newcommand{\Q}{\mathbb{Q}}   % Rational numbers
\newcommand{\R}{\mathbb{R}}   % Real numbers
\newcommand{\C}{\mathbb{C}}   % Complex numbers
\newcommand{\A}{\mathbb{A}}   % Affine space
\renewcommand{\P}{\mathbb{P}} % Projective space


%% New commands for vectors
\renewcommand{\a}{\mathbf{a}}
\renewcommand{\b}{\mathbf{b}}
\renewcommand{\c}{\mathbf{c}}
\renewcommand{\v}{\mathbf{v}}
\newcommand{\w}{\mathbf{w}}
\newcommand{\x}{\mathbf{x}}
\newcommand{\y}{\mathbf{y}}
\newcommand{\z}{\mathbf{z}}
\newcommand{\0}{\mathbf{0}}
\newcommand{\1}{\mathbf{1}}


%% Miscellaneous
\renewcommand{\qedsymbol}{\(\blacksquare\)}


\title{Fire Detection Using MODIS}
\author{Paulina Tedesco}
%\supervisor{}
% Multiple supervisors: \supervisor{Supervisor 1}{Supervisor 2}...{Supervisor n}
% Skip supervisor for MAT2500


\begin{document}


\projectfrontpage


\begin{abstract}
    \noindent
    Brief summary of the paper.
\end{abstract}

% ------------------------------------------------------------------------------------------------------
\chapter{Introduction}

Wildfires are unplanned fires that occur in natural areas such as forests, grasslands, or prairies, causing a tremendous impact on the environment. They pose a threat to natural resources, properties, human health, wildlife, ecosystems, weather, and climate. 

Wildfire activity is strongly influenced by climate and weather conditions, and as global temperatures, droughts, and extremes increase, there appears to be an increasing trend in fire activity (\cite{jolly_and_william_2015}, IPCC). Two crucial factors that affect fires are high temperatures and low humidity. The main natural cause of fires is lightning, which is related to deep convective systems favored by warm conditions (\cite{veraverbeke_et_al_2017}). However, hotter and dryer conditions also set the stage for fires originated by human activity. Another consequence of a warmer climate is that nighttime temperatures are higher, allowing fires to burn at night and extending over more days, where, previously, cooler temperatures at night would weaken or extinguish the fires.

Fire feedbacks are complex mechanisms; fires can, directly and indirectly, increase carbon emissions to the atmosphere. While burning, they release carbon from trees or the soil. In addition, dead trees release carbon as they decompose and can no longer act as a carbon sink. There is also a mixed effect of aerosols on climate due to fires: on the one hand, dark aerosols (black carbon) absorb heat from the sunlight in the atmosphere and darken the surfaces contributing to its melt, while on the other hand, light-colored aerosols may have the opposite effect. Aerosols may also make it harder for water droplets to form in the tropics (\cite{tosca_et_al_2012}). Overall, fire feedback mechanisms have an impact on both local temperatures and the water cycle.

Understanding the immediate and long-term effects of wildfires requires global datasets, such as satellite data, that allow us to detect fires, map their burned areas, and trace the smoke in the atmosphere.



\begin{itemize}
    \item  Wildfire is a major natural disturbance that has
tremendous impact on environment, humans and wildlife,
ecosystem, weather, and climate. There appears to be an
increasing trend of natural fire activity [Weber and Stocks,
1998] that coincides with the observed and predicted
climate-warming trend in mid- and high latitudes [International Panel on Climate Change (IPCC), 1990;
    \item Fires depend on weather conditions
    \item 2018 was exceptionally warm and dry in Scandinavia
    \item something about MODIS - LST
\end{itemize}

Purpose of the paper, historical context, necessary background information and notation.

% ------------------------------------------------------------------------------------------------------
\chapter{Methods and Data}


Full proofs, numerical implementations.
Remember to cite your sources,
such as \cite{Hel17}.

\begin{theorem}[Pythagoras]
    \label{thm:pythagoras}
    In a right triangle,
    the square of the hypotenuse is equal to the sum of the squares of the other two sides.
    That is,
    \begin{equation}
        \label{eq:pythagoras}
            a^2 + b^2 = c^2,
    \end{equation}
    where \(c\) is the length of the hypotenuse and \(a\) and \(b\) are the lengths of the two other sides.
\end{theorem}

\begin{proof}
    Draw a figure.
\end{proof}

% ------------------------------------------------------------------------------------------------------
\chapter{Analysis}

% ------------------------------------------------------------------------------------------------------
\chapter{Conclusions}


Optional. Results, consequences, future work.

\cref{tab:numbers} lists some integers satisfying \cref{eq:pythagoras} of \cref{thm:pythagoras}.

\begin{table}[htbp]
    \centering
    \begin{tabular}{@{}ccc@{}}
        \toprule
        \(\boldsymbol{a}\) & \(\boldsymbol{b}\) & \(\boldsymbol{c}\)
        \\
        \midrule
        3 & 4 & 5
        \\
        65 & 72 & 97
        \\
        \bottomrule
    \end{tabular}
    \caption{Some interesting numbers}
    \label{tab:numbers}
\end{table}


\printbibliography


\end{document}